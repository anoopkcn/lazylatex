%example.tex

\documentclass[12pt,a4paper]{report}
\usepackage[utf8]{inputenc}
\usepackage{graphicx}
\usepackage{lazylatex}
\usepackage{lipsum}
\usepackage{amsmath}

\begin{document}
%
\tableofcontents

\section{Introduction}
\lipsum[1]
\section{Packages Used}
\section{Structural Elements}
Uses native LaTeX structural elements(Section, Subsection, etc,.). Paragraphs are indented even the first paragraph in each section, this can be easily turned off using latex native commands.
\section{Paragraph Level Markup}
\subsection{Inline Markup}
Documents contain text and may contain inline markup: \pre{inline literals}, \emph{emphasis}, \textbf{strong emphasis}, stand alone links (\url{https://www.wikibooks.org}), external hyperlinks (\href{https://www.wikibooks.org}{wikibooks home}), and \guilabel{some action}. For software library or a package use \lib{PackageName} and a \fname{filename.out}
\section{Lists \& Tables}

\lipsum[1]

\indentBlock{}{is is an indent block tas. Mauris ut leo. Cras viverra this is an indent block tas. Mauris ut leo. Cras viverra metus rhoncus sem. Nulla et lectus vestibulum urna fringilla ultrices. Phasellus eu tellus sit}

\indentHang{this is an indent block tas. Mauris ut leo. Cras viverra metus rhoncus sem. Nulla et lectus vestibulum urna fringilla ultrices. Phasellus eu tellus sit amet tortor gravida placerat. Integer sapien est, iaculis in, preti}

\subsection{code format}

%
\begin{code}{go}
package main

import "fmt"

func main() {
  var x int = 7
  fmt.Println(x)
  fmt.Println(&x)
}
\end{code}
%
If we run the above program
\begin{literal}
>> go run pointers.go
7
0xc00001c0a8
\end{literal}
%
Line two is the value of the variable x (i.e. 7) and line three is the address where it is stored (i.e. 0xc00001c0a8)
\begin{equation}
  e^{i\pi}+1=0
  \label{eqn:euler}
\end{equation}

\subsection{Pointers}
A pointer is a special kind of datatype which takes a reference as its value . It is defined in the program as *datatype (i.e. \texttt{*int, *uint, *char, *byte}, etc,) they declare a \pre{pointer} object which points to something(variables, constants, etc.) with a given datatype. Code \nameref{myCodeLabel}
In the program given below line 9 and 10 prints the same address
%
\begin{code}{go}
package main

import "fmt"

func main() {
  var x int = 73
  var xptr *int = &x
  fmt.Println(x)
  fmt.Println(&x)
  fmt.Println(xptr)
}
\end{code}
%
Running the above program
\begin{literal}
>> go run pointers.go
73
0xc00001c0a8
0xc00001c0a8
\end{literal}
%
As one can notice line three and four are the same. In order to get the value of what the pointer is pointing at one can use the \pre{*} operator. This operator operates from the left of the pointer variable. Operating on the right has no meaning if the variable is a pointer and if the variable is any-other datatype then it has the usual meaning of multiplication. Since \pre{*} is doing the opposite of the referencing operator \pre{\&} it is called \pre{dereference} operator. The following example illustrates the operation.
%
\begin{note}
	Not to be confused with multiplication operator \texttt{*} or the asterisk in front of the pointer data type \texttt{*datatype}
\end{note}
\begin{note}[My Note Titile]
	Not to be confused with multiplication operator \texttt{*} or the asterisk in front of the pointer data type \texttt{*datatype}
\end{note}
%
\begin{lazycode}[code title,label={myCodeLabel},nameref={code title}]{go}
package main

import "fmt"

func main() {
  var x int = 73
  var xptr *int = &x
  fmt.Println(x)
  fmt.Println(&x)
  fmt.Println(xptr)
  fmt.Println(*xptr)
}
\end{lazycode}
%
Result:
%
\begin{literal}
>> go run pointers.go
73
0xc00001c0a8
0xc00001c0a8
73
\end{literal}

\begin{warning}[My Warning Titile]
\begin{literal}
  #leave out the optional title for default.
\end{literal}
this is a warning
\end{warning}
\begin{tip}{this is a tiphint}\end{tip}
\begin{error}this is an error\end{error}

Sed ut \textbf{Table} ~\ref{mytable} unde \textbf{Table} \ref{mytable2} omnis iste natus error sit voluptatem accusantium doloremque laudantium, totam rem aperiam, eaque ipsa quae ab illo inventore veritatis et quasi architecto beatae vitae dicta sunt explicabo
%
\begin{wrapfigure}{R}{0.6\textwidth}
	\begin{sidebar}{My Own Heading}
		Nam dui ligula, fringilla a, euismo d so dales, sollicitudin vel, wisi. Morbi auctor lorem non justo. Nam lacus lib ero, pretium at, lob ortis vitae, ul- tricies et, tellus. Donec aliquet, tortor sed accumsan bib endum, erat ligula aliquet magna, vitae ornare o dio metus a mi. Morbi ac orci et nisl hendrerit mollis. Susp endisse
	\end{sidebar}
\end{wrapfigure}
%
Nemo enim ipsam voluptatem quia voluptas sit aspernatur aut odit aut fugit, sed quia consequuntur magni dolores eos qui ratione voluptatem sequi nesciunt. Neque porro quisquam est, qui dolorem ipsum quia dolor sit amet, consectetur, adipisci velit, sed quia non numquam eius modi tempora incidunt ut labore et dolore magnam aliquam quaerat voluptatem. Ut enim ad minima veniam, quis nostrum exercitationem ullam corporis suscipit laboriosam, nisi ut aliquid ex ea commodi consequatur? Quis autem vel eum iure reprehenderit \lib{helloMKL}

qui in ea voluptate velit esse quam nihil molestiae consequatur, vel illum qui dolorem eum fugiat quo voluptas nulla pariatur? At vero eos et accusamus et iusto odio dignissimos ducimus qui blanditiis praesentium voluptatum deleniti atque corrupti quos dolores et quas molestias excepturi sint occaecati cupiditate non provident, similique sunt in culpa qui officia deserunt mollitia animi, id est laborum et dolorum fuga. Et harum quidem rerum facilis est et expedita distinctio. Nam libero tempore, cum soluta nobis est eligendi optio cumque nihil impedit quo minus id quod maxime placeat facere
%
\subsection{Tables}
%
Nemo enim ipsam voluptatem quia voluptas sit aspernatur aut odit aut fugit, sed quia consequuntur magni dolores eos qui ratione voluptatem sequi nesciunt. Neque porro quisquam est, qui dolorem ipsum quia dolor sit amet, consectetur, adipisci velit, sed quia non numquam eius modi tempora incidunt ut labore et dolore magnam aliquam quaerat voluptatem. Ut enim ad minima veniam, quis nostrum exercitationem ullam corporis suscipit laboriosam, nisi ut aliquid ex ea commodi consequatur? Quis autem vel eum iure reprehenderit qui in ea voluptate velit esse quam nihil molestiae consequatur, vel illum qui dolorem eum fugiat quo voluptas nulla pariatur? At vero eos et accusamus et iusto odio dignissimos ducimus \\

\begin{table}[h!]
	\begin{lazytable}{X||Y|Y|Y|Y|Y}
	\bf Group & \bf One     & \bf Two     & \bf Three    & \bf Four     & \bf Sum\\
	\hline
	\hline
	Red   & 1000.00 & 2000.00 &  3000.00 &  4000.00 & 10000.00\\
	\hline
	Green & 2000.00 & 3000.00 &  4000.00 &  5000.00 & 14000.00\\
	\hline
	Blue  & 3000.00 & 4000.00 &  5000.00 &  6000.00 & 18000.00\\
	\hline
	Sum   & 6000.00 & 9000.00 & 12000.00 & 15000.00 & 42000.00
	\end{lazytable}
	\caption{this is a table}
	\label{mytable}
\end{table}
\lipsum[2]

\indentBlock{}{
    this is a Block Quote . Nam lacus libero, pretium at, lobortis vitae, ultricies et, tellus. Donec aliquet, tortor sed accumsan bibendum, erat ligula aliquet magna, vitae ornare odio metus a mi. Morbi ac orci et nisl hendrerit mollis. Suspendisse ut massa. Cras nec ante. Pellentesque a nulla. Cum sociis natoque pe
}

\begin{table}[h!]
	\begin{lazytable}{Z|Z|Z|Z|Z}
	\bf Group & \bf One     & \bf Two     & \bf Three     & \bf Sum\\
	\hline
	\hline
	Red & 1000.00 & 2000.00 &  3000.00  & 10000.00\\
	\hline
	Green & 2000.00 & 3000.00 &  4000.00  & 14000.00\\
	\hline
	Blue  & 3000.00 & 4000.00 &  5000.00  & 18000.00\\
	\hline
	Sum   & 6000.00 & 9000.00 & 12000.00  & 42000.00\\
	\hline
	Blue  & 3000.00 & 4000.00 &  5000.00  & 18000.00\\
	\hline
	Gray  & 3000.00 & 4000.00 &  5000.00  & 18000.00\\
	\end{lazytable}
	\caption{this is a table2}
	\label{mytable2}
\end{table}


\end{document}
