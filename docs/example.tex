%example.tex

\documentclass[12pt,a4paper]{report}
\usepackage[utf8]{inputenc}
\usepackage{graphicx}
\usepackage{lazylatex}
\usepackage{lipsum}
\usepackage{amsmath}
\tcbuselibrary{documentation} %tcbuselibrary comes from lazylatex

\begin{document}
\tableofcontents

\section{Introduction}
\lib{lazylatex} is a {\LaTeX} package inspired by sphinx-rtd-theme. Build with {\LaTeX} packages such as tcolorbox, tikz, etc,. Not all text elements are a simulation of rtd-theme, it also adopts text element style from different sources. Please consult \fname{example.tex} source file of this \emph{pdf} for usage and examples of some of the commands.
\section{Packages Used}
\begin{itemize}
	\item \lib{geometry} - for page dimension setup
	\item \lib{xcolor} - for better colour definitions
	\item \lib{tcolorbox} - for inline markup and admonitions
	\item \lib{tikz} - additional tex features(rotation and placing) also for graphics
	\item \lib{warapfig} - for floating area(size and place) when creating sidebar
	\item \lib{tabularx}, \lib{array}, \lib{colortbl} - fot styling tables
	\item \lib{hyperref} - for link styling and viewer setup.
	\item \lib{setspace} - for line spacing and indented blocks
\end{itemize}

\section{Structural Elements}
Uses native LaTeX structural elements(Section, Subsection, etc,.). Paragraphs are indented even the first paragraph in each section, this can be easily turned off using \pre{setlength} latex command
\section{Paragraph Level Markup}

\subsection{Inline Markup}
Documents contain text and may contain inline markup: \pre{inline literals}, \emph{emphasis}, \textbf{strong emphasis}, stand alone links (\url{https://www.wikibooks.org}), external hyperlinks (\href{https://www.wikibooks.org}{wikibooks home}),internal reference (Eq:\ref{eqn:euler}, Code:\ref{myCodeLabel} with \nameref{myCodeLabel}) and \guilabel{some action}. For software library or a package use \lib{PackageName} and file names are indicated with \fname{filename.ext}

\subsection{Math}
As usual,
\begin{equation}
  e^{i\pi}+1=0
  \label{eqn:euler}
\end{equation}

\subsection{Line Blocks}
Line blocks are styled in two different ways, \emph{indent blocks} and \emph{hanging blocks}. The former takes an optional number in units of \emph{cm}, a measure of how much indentation is needed. Examples:\\

\subsubsection{Indent Block}
\begin{docCommand}%
	[]{indentBlock}{\oarg{optional left width}\marg{content}}\tcbdocmarginnote{syntax}
	Where \meta{optional left width} in \emph{cm} is the width with which the block is shifted to right.
\end{docCommand}
\noindent
\textbf{Example:}	

\indentBlock{}{is is an indent block tas. Mauris ut leo. Cras viverra this is an indent block tas. Mauris ut leo. Cras viverra metus rhoncus sem. Nulla et lectus vestibulum urna fringilla ultrices. Phasellus eu tellus sit}
\subsubsection{Hanging Block}
\begin{docCommand}%
	[]{indentHang}{\marg{content}}
	Where \meta{conent} is the text.
\end{docCommand}
\noindent
\textbf{Example:}

\indentHang{this is an indent block tas. Mauris ut leo. Cras viverra metus rhoncus sem. Nulla et lectus vestibulum urna fringilla ultrices. Phasellus eu tellus sit amet tortor gravida placerat. Integer sapien est, iaculis in, preti}\\

\begin{note}
Both \cs{indentBlock} and \cs{indentHang} commands must preceed a blank line or new line \LaTeX\ character.
\end{note}


\subsection{Code Blocks}
code blocks are enabled using \lib{listings} package. A version of the \pre{github-light} theme is replicated for the code blocks. Examples:

\noindent
\begin{lstlisting}[language=go]
package main

import "fmt"

func main() {
  var x int = 73
  var xptr *int = &x
  fmt.Println(x)
  fmt.Println(&x)
  fmt.Println(xptr)
  fmt.Println(*xptr)
}
\end{lstlisting}

\subsection{Admonitions}
There are four admonitions \pre{note}, \pre{tip}, \pre{warning} and \pre{error}. They all take one optional argument which is the title of the admonition. If not given they have a default value of \textbf{!Note}, \textbf{!Tip}, \textbf{!Warning} and \textbf{!Error} respectively. Examples:\\

\begin{note}
	Lorem ipsum dolor sit amet, consectetuer adipiscing elit. Ut purus elit, vestibulum ut, placerat ac, adipiscing vitae, felis. Curabitur dictum gravida mauris. Nam arcu lib ero, nonummy eget, consectetuer id, vulputate a, magna. Donec vehicula augue eu neque. Pellentesque habitant morbi tristique senectus et netus et malesuada fames ac turpis egestas. Mauris ut leo. Cras viverra metus rhoncus sem. Nulla et lectus vestibulum urna fringilla ultrices.
\end{note}
\begin{tip}
Lorem ipsum dolor sit amet, consectetuer adipiscing elit. Ut purus elit, vestibulum ut, placerat ac, adipiscing vitae, felis. Curabitur dictum gravida mauris. Nam arcu lib ero, nonummy eget, consectetuer id, vulputate a, magna. Donec vehicula augue eu neque. Pellentesque habitant morbi tristique senectus et netus et malesuada fames ac turpis egestas. Mauris ut leo. Cras viverra metus rhoncus sem.
\end{tip}
\begin{warning}
Lorem ipsum dolor sit amet, consectetuer adipiscing elit. Ut purus elit, vestibulum ut, placerat ac, adipiscing vitae, felis. Curabitur dictum gravida mauris. Nam arcu lib ero, nonummy eget, consectetuer id, vulputate a, magna. Donec vehicula augue eu neque. 
\end{warning}
\begin{error}
Lorem ipsum dolor sit amet, consectetuer adipiscing elit. Ut purus elit, vestibulum ut, placerat ac, adipiscing vitae, felis. Curabitur dictum gravida mauris. Nam arcu lib ero, nonummy eget, consectetuer id, vulputate a, magna. Donec vehicula augue eu neque.
\end{error}
\begin{note}[Custom Title]
	Nam arcu lib ero, nonummy eget, consectetuer id, vulputate a, magna. Donec vehicula augue eu neque. Pellentesque habitant morbi tristique senectus et netus et malesuada fames ac turpis egestas. Mauris ut leo. Cras viverra metus rhoncus sem. Nulla et lectus vestibulum urna fringilla ultrices.
\end{note}
\begin{tip}[This Hint is cool]
Donec vehicula augue eu neque. Pellentesque habitant morbi tristique senectus et netus et malesuada fames ac turpis egestas. Mauris ut leo. Cras viverra metus rhoncus sem.
\end{tip}

\subsection{Sidebar}
The \pre{sidebar} feature is rendered with \lib{wrapfig} latex package for placing it wherever in the latex document also for the witdth of the sidebar. Adding caption and label is also possible. Text will wrap around the sidebar, \textbf{Example}:\\

\lipsum[1]
\begin{wrapfigure}{r!}{0.6\textwidth}
	\begin{lsidebar}{My Heading}
		Lorem ipsum dolor sit amet, consectetuer adipiscing elit. Ut purus elit, vestibulum ut, placerat ac, adipiscing vitae, felis. Curabitur dictum gravida mauris. Nam arcu lib ero, nonummy eget, consectetuer id, vulputate a, magna. Donec vehicula augue eu neque. Pellentesque habitantmorbi tristique senectus et netus et malesuada fames ac turpis egestas.
	\end{lsidebar}
\end{wrapfigure}
\lipsum[2]

\section{Lists \& Tables}
\label{listntables}
Numbered and bulleted lists work pretty well in latex but it doesn't provide options for \textbf{Field lists} and \textbf{Hlists}. \lib{lazylatex} provide \pre{llist} environment to create Field Lists and Hlists. Table is created with \pre{lazytable}. Both environments take 2 optional arguments, first \oarg{optional width} which is the width of the list or table in a page and second \marg{alignment}, the alignment of the elements in a table or list.\\
\begin{tip}[Command]
\begin{docEnvironment}%
	[doclang/environment content=content]% 
	{llist or lazytable}{\oarg{optional width}\marg{alignment}}
	Where:\\
	\oarg{optional width} is specified in the units of \cs{pagewidth}\\
	\noindent	
 	\marg{alignment} is specified using a combination of \pre{X},\pre{Y},\pre{Z}, \pre{L\marg{column width}}, \pre{R\marg{column width}}, \pre{C\marg{column width}}.\\
 	\noindent
	\marg{column width} option for \pre{L,R,C} are specified with a real number indicating the width each column must take within the specified \oarg{optional width}\\
	\noindent
	\pre{X} and \pre{L} - align the column content to left, \pre{Y} and \pre{R} aligns the content to right and \pre{Z} and \pre{C} aligns the content to center
\end{docEnvironment}
\end{tip}
	
\subsection{Native {\LaTeX} List}
\begin{itemize}
	\item Item A
	\item Item B
	\item Item C	
\end{itemize}

\subsection{Hlists}
\begin{tip}[Example]
\begin{docEnvironment}%
	[doclang/environment content=content]% 
	{llist}{\oarg{optional width}\marg{alignment}}
	Where \meta{content} is a set of \cs{litem}  separated by \&. \oarg{optional width} is [0.5\cs{textwidth}] and \marg{alignment} is \brackets{XXX}\\
	Refer Section ~\ref{listntables} for values each option can take.
\end{docEnvironment}
\end{tip}
\bigskip

\begin{llist}[0.5\textwidth]{XXX}
  \litem{item 00} & \litem{item 01} & \litem{item 02}\\
  \litem{item 10}  & \litem{item 11} & \litem{item 12}\\
  \litem{item 20}  & \litem{item 21} & \litem{item 22}\\
\end{llist}

\subsection{Field List}
\begin{tip}[Example]
\begin{docEnvironment}%
	[doclang/environment content=content]% 
	{llist}{\oarg{optional width}\marg{alignment}}
	Where \meta{content} is a set of text  separated by \&. \oarg{optional width} is default i.e \cs{textwidth} and \marg{alignment} is \brackets{L\brackets{0.35}X}\\
	Refer Section ~\ref{listntables} for values each option can take.
\end{docEnvironment}
\end{tip}
\bigskip

\begin{llist}{L{0.35}X}
	\textbf{Author:} &	Anoop Chandran \\
	\textbf{Address:} & 123 Example Street, Example, Example \\
	\textbf{Contact:}	& strivetobelazy@gmail.com \\
	\textbf{Authors:} & Me; Myself; I \\
	\textbf{Organization:} & humankind \\
	\textbf{Date:} & \today \\
	\textbf{Status:}	& This is a ``work in progress" \\
	\textbf{Revision:} &	Revision: 100001  \\
	\textbf{Version:} & 0.01 \\
	\textbf{Copyright:}	& This document has been placed in the public domain. You may do with it as you wish.
 \end{llist}

\subsection{Tables}
Three examples are shown for Tables however one can combine the features of all the tables in different ways.
\begin{tip}[Example 1]
\begin{docEnvironment}%
	[doclang/environment content=content]% 
	{lazytable}{\oarg{optional width}\marg{alignment}}
	Where \meta{content} is a set of text  separated by \&. \oarg{optional width} is default i.e \cs{textwidth} and \marg{alignment} is \brackets{L\brackets{1}$\vert$ R\brackets{0.5}$\vert$ R\brackets{0.5}$\vert$ C\brackets{2}}, argument sum, $1+0.5+0.5+2=$num columns\\
	Refer Section ~\ref{listntables} for values each option can take.
\end{docEnvironment}
\end{tip}
\bigskip

\tcbdocmarginnote{variable width columns}
\begin{ltable}{L{1}|R{0.5}|R{0.5}|C{2}}
  label 00 & label 01 & label 02 & label 03 \\
  item 10  & item 11  & item 12  & item 13  \\
  item 20  & item 21  & item 22  & item 23  \\
\end{ltable}

The following examples have captions, to do this one has to wrap \pre{lazytable} inside a \pre{table} environment. Caption and lablel would be the contents of this table but outside \pre{lazytable}\\

\begin{tip}[Example 2]
\begin{docEnvironment}%
	[doclang/environment content=content]% 
	{lazytable}{\oarg{optional width}\marg{alignment}}
	\oarg{optional width} is default i.e \cs{textwidth} and \marg{alignment} is \brackets{X$\vert$$\vert$Y$\vert$Y$\vert$Y$\vert$Y$\vert$}
	Refer Section ~\ref{listntables} for values each option can take.
\end{docEnvironment}
\end{tip}
\bigskip

\tcbdocmarginnote{fixed width columns}
\begin{table}[h!]
	\begin{ltable}{X||Y|Y|Y|Y|Y}
	\bf Group & \bf One     & \bf Two     & \bf Three    & \bf Four     & \bf Sum\\
	\hline
	\hline
	Red   & 1000.00 & 2000.00 &  3000.00 &  4000.00 & 10000.00\\
	\hline
	Green & 2000.00 & 3000.00 &  4000.00 &  5000.00 & 14000.00\\
	\hline
	Blue  & 3000.00 & 4000.00 &  5000.00 &  6000.00 & 18000.00\\
	\hline
	Sum   & 6000.00 & 9000.00 & 12000.00 & 15000.00 & 42000.00
	\end{ltable}
	\caption{this is a table}
	\label{mytable}
\end{table}
To align a table with reduced with at the center of the page, wrap the \pre{lazytable} inside a \pre{center} environment.\\

\begin{tip}[Example 3]
\begin{docEnvironment}%
	[doclang/environment content=content]% 
	{lazytable}{\oarg{optional width}\marg{alignment}}
	\oarg{optional width} is  0.6\cs{textwidth} and \marg{alignment} is \brackets{Z$\vert$Z$\vert$Z$\vert$Z}\\
	Refer Section ~\ref{listntables} for values each option can take.
\end{docEnvironment}
\end{tip}
\bigskip

\begin{table}[h!]
\begin{center}
	\begin{ltable}[0.6\textwidth]{Z|Z|Z|Z}
	\bf Group & \bf One     & \bf Two     & \bf Three\\
	\hline
	\hline
	Red & 1000.00 & 2000.00   & 10000.00\\
	\hline
	Green & 2000.00 & 3000.00   & 14000.00\\
	\hline
	Blue  & 3000.00 & 4000.00   & 18000.00\\
	\hline
	Sum   & 6000.00 & 9000.00   & 42000.00\\
	\hline
	Blue  & 3000.00 & 4000.00   & 18000.00\\
	\hline
	Gray  & 3000.00 & 4000.00  & 18000.00\\
	\end{ltable}
	\caption{This is a table-2}
	\label{mytable2}
\end{center}
\end{table}
%
\noindent
\textbf{Example 4:}\\

\begin{table}[h!]
	\begin{ltable}{L{1.25}|C{0.75}|C{0.5}|C{0.5}}
		\bf Header row,(header rows optional) & \bf Header 2 & \bf Header 3     & \bf Header 4\\
		\hline\hline
		body row 1, column 1 & column 2 & column 3   & column 4\\
		\hline
		body row 2 & \multicolumn{3}{c}{Cells may span columns}\\
		\hline
		body row 3  & & & \\
		\hline
		body row 4  & & & \\
		\hline
		body row 5  & \multicolumn{3}{c}{Cells may also be empty}\\
	\end{ltable}
	\caption{This is a table-3}
	\label{mytable2}
\end{table}

\end{document}
